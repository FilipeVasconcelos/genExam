\documentclass[11pt,a4paper,twoside]{article} %twocolumn
\usepackage[francais]{babel}
\usepackage[babel=true,kerning=true]{microtype}
\usepackage[utf8]{inputenc}
\usepackage[T1]{fontenc}
\usepackage{lmodern}
\usepackage{geometry}
\geometry{hmargin=2.5cm,vmargin=2.5cm}
\usepackage{graphicx}
\usepackage[dvipsnames]{xcolor}
\usepackage{pifont}% http://ctan.org/pkg/pifont
\newcommand{\cmark}{\ding{51}}%
\newcommand{\xmark}{\ding{55}}%
\graphicspath{{./img/}}
\usepackage{examen}
%%%%%%%%%%%%%%%%%%%%%%%%%%%%%%%%%%%%%%%%%%%%%%%%%%%%%%%%%%%%%%%%%%%%%%%%%%%%%%%%
%Modifier les variables suivantes 
\newcommand\promo{INGE1}                              %ex. IngéSUP, IngéSPE, Ingé1
\newcommand\module{UNIX}                   %ex. Systèmes Techniques, Mathématiques Fondamentales
\newcommand\annee{XX janvier 202X}                   %ex. 2022-2023
\newcommand\epreuve{Examen final}                 %ex. MidTerm, FinalExam, Rattrapage
\newcommand\titreEval{} %ex. Dynamique et Déformation
\newcommand\dureeEval{2}                     %en nombre d'heures
\reponse{true}
%%%%%%%%%%%%%%%%%%%%%%%%%%%%%%%%%%%%%%%%%%%%%%%%%%%%%%%%%%%%%%%%%%%%%%%%%%%%%%%%
%%%%%%%%%%%%%%%%%%%%%%%%%%%%%%%%%%%%%%%%%%%%%%%%%%%%%%%%%%%%%%%%%%%%%%%%%%%%%%%%
\begin{document}
\maketitle
%%%%%%%%%%%%%%%%%%%%%%%%%%%%%%%%%%%%%%%%%%%%%%%%%%%%%%%%%%%%%%%%%%%%%%%%%%%%%%%%
\thispagestyle{fancy}
%%%%%%%%%%%%%%%%%%%%%%%%%%%%%%%%%%%%%%%%%%%%%%%%%%%%%%%%%%%%%%%%%%%%%%%%%%%%%%%%
\newpage

\section*{Consignes}
\begin{itemize}
    \item Dans le cas des questions à choix multiples : veuillez répondre en entourant la ou les bonnes réponses. 
    \item Dans le cas des questions d'appariement : tracez un trait entre les énoncés et leurs réponses.
\end{itemize}
\section*{Questions}
\question{Associer chaque script à une tâche }\\[0.2cm]
\indent\resizebox{\linewidth}{!}{%
\begin{tikzpicture}

    \node[text width=39.34398mm,align=left] (a1) at (0,0) {\textbf{ A. } \texttt{aleaQ}     };
    \node (b1) at ($(a1.east)+(1,0)$) {};
    \node (c1) at ($(b1.east)+(2,0)$) {};
    \draw[thick,outer sep=8mm] (b1) circle(0.5ex);
    \draw[thick,outer sep=8mm] (c1) circle(0.5ex);
    \node[text width=111.97902mm,align = right] (d1) at ($(c1.east)+(6,0)$) {\textbf{ 1. } Mélange les propositions d'une question passée en paramètre  };

    \node[text width=39.34398mm,below=0.75cm of a1.center,anchor=center,align=left] (a2) {\textbf{ B. } \texttt{getTAGS}   };
    \node[below=0.75cm of b1.center,anchor=center] (b2) {};
    \node[below=0.75cm of c1.center,anchor=center] (c2) {};
    \draw[thick,outer sep=8mm] (b2) circle(0.5ex);
    \draw[thick,outer sep=8mm] (c2) circle(0.5ex);
    \node[below=0.75cm of d1.center,anchor=center,text width=111.97902mm,align = right] (d2) {\textbf{ 2. } Génère un document pdf d'examen à partir d'une banque de questions };

    \node[text width=39.34398mm,below=0.75cm of a2.center,anchor=center,align=left] (a3) {\textbf{ C. } \texttt{getQ}      };
    \node[below=0.75cm of b2.center,anchor=center] (b3) {};
    \node[below=0.75cm of c2.center,anchor=center] (c3) {};
    \draw[thick,outer sep=8mm] (b3) circle(0.5ex);
    \draw[thick,outer sep=8mm] (c3) circle(0.5ex);
    \node[below=0.75cm of d2.center,anchor=center,text width=111.97902mm,align = right] (d3) {\textbf{ 3. } Renvoie la chaine de la question passée en paramètre };

    \node[text width=39.34398mm,below=0.75cm of a3.center,anchor=center,align=left] (a4) {\textbf{ D. } \texttt{genExam}   };
    \node[below=0.75cm of b3.center,anchor=center] (b4) {};
    \node[below=0.75cm of c3.center,anchor=center] (c4) {};
    \draw[thick,outer sep=8mm] (b4) circle(0.5ex);
    \draw[thick,outer sep=8mm] (c4) circle(0.5ex);
    \node[below=0.75cm of d3.center,anchor=center,text width=111.97902mm,align = right] (d4) {\textbf{ 4. } Extrait la ligne des TAGS d'une question passée en paramètre };

\end{tikzpicture}
}%
\question{Quel est le but du script \texttt{genExam} ?}\\[0.2cm]
\indent\textbf{A.}  Générer un document pdf d'examen à partir d'une banque de questions \newline\hfill
\indent\textbf{B.}  Un prétexte pour faire du bash \newline\hfill
\indent\textbf{C.}  Faire le café \newline\hfill
\indent\textbf{D.}  Permet de ne pas aller en cours et de lancer un script le jour de l'examen \newline\hfill
\end{document}
