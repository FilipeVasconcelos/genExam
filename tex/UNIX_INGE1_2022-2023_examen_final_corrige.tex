\documentclass[11pt,a4paper,twoside]{article} %twocolumn
\usepackage[francais]{babel}
\usepackage[babel=true,kerning=true]{microtype}
\usepackage[utf8]{inputenc}
\usepackage[T1]{fontenc}
\usepackage{lmodern}
\usepackage[top=2cm, bottom=2.5cm, left=2.5cm, right=2.5cm]{geometry}
\usepackage{graphicx}
\usepackage[dvipsnames]{xcolor}
\usepackage{pifont}% http://ctan.org/pkg/pifont
\newcommand{\cmark}{\ding{51}}%
\newcommand{\xmark}{\ding{55}}%
\graphicspath{{./img/}}
\usepackage{examen}
%%%%%%%%%%%%%%%%%%%%%%%%%%%%%%%%%%%%%%%%%%%%%%%%%%%%%%%%%%%%%%%%%%%%%%%%%%%%%%%%
%Modifier les variables suivantes 
\promo{INGE1}     % ex. IngéSUP, IngéSPE, Ingé1
\module{UNIX}   % ex. Systèmes Techniques, Mathématiques Fondamentales
\annee{XX janvier 202X}      % ex. 2022-2023
\epreuve{Examen final}  % ex. MidTerm, FinalExam, Rattrapage
\titreEval{None} % ex. Dynamique et Déformation
\dureeEval{2} % ex. 2 (en nombre d'heures)
\logo{}               %Image du logo
\esme{false}          %examen ESME true or false
\reponse{true}        %document réponse true or false
%%%%%%%%%%%%%%%%%%%%%%%%%%%%%%%%%%%%%%%%%%%%%%%%%%%%%%%%%%%%%%%%%%%%%%%%%%%%%%%%
%%%%%%%%%%%%%%%%%%%%%%%%%%%%%%%%%%%%%%%%%%%%%%%%%%%%%%%%%%%%%%%%%%%%%%%%%%%%%%%%
\begin{document}
\maketitle
%%%%%%%%%%%%%%%%%%%%%%%%%%%%%%%%%%%%%%%%%%%%%%%%%%%%%%%%%%%%%%%%%%%%%%%%%%%%%%%%
\thispagestyle{fancy}
%%%%%%%%%%%%%%%%%%%%%%%%%%%%%%%%%%%%%%%%%%%%%%%%%%%%%%%%%%%%%%%%%%%%%%%%%%%%%%%%
\newpage

\section*{Consignes}
\begin{itemize}
    \item Dans le cas des questions à choix multiples : veuillez répondre en entourant la ou les bonnes réponses. 
    \item Dans le cas des questions d'appariement : tracez un trait entre les énoncés et leurs réponses.
\end{itemize}
\section*{Questions}
\question{Associer chaque script à une tâche }\\[0.0cm]
\indent\resizebox{\linewidth}{!}{%
\begin{tikzpicture}

    \node[text width=45.3969mm,align=left] (a1) at (0,0) {\textbf{ A. } \texttt{getQ}      };
    \node (b1) at ($(a1.east)+(1,0)$) {};
    \node (c1) at ($(b1.east)+(2,0)$) {};
    \draw[thick,outer sep=8mm] (b1) circle(0.5ex);
    \draw[thick,outer sep=8mm] (c1) circle(0.5ex);
    \node[text width=118.03194mm,align = right] (d1) at ($(c1.east)+(6,0)$) {\textbf{ 1. } Mélange les propositions d'une question passée en paramètre  };

    \node[text width=45.3969mm,below=0.75cm of a1.center,anchor=center,align=left] (a2) {\textbf{ B. } \texttt{getTAGS}   };
    \node[below=0.75cm of b1.center,anchor=center] (b2) {};
    \node[below=0.75cm of c1.center,anchor=center] (c2) {};
    \draw[thick,outer sep=8mm] (b2) circle(0.5ex);
    \draw[thick,outer sep=8mm] (c2) circle(0.5ex);
    \node[below=0.75cm of d1.center,anchor=center,text width=118.03194mm,align = right] (d2) {\textbf{ 2. } Renvoie la chaine de la question passée en paramètre };

    \node[text width=45.3969mm,below=0.75cm of a2.center,anchor=center,align=left] (a3) {\textbf{ C. } \texttt{genExam}   };
    \node[below=0.75cm of b2.center,anchor=center] (b3) {};
    \node[below=0.75cm of c2.center,anchor=center] (c3) {};
    \draw[thick,outer sep=8mm] (b3) circle(0.5ex);
    \draw[thick,outer sep=8mm] (c3) circle(0.5ex);
    \node[below=0.75cm of d2.center,anchor=center,text width=118.03194mm,align = right] (d3) {\textbf{ 3. } Extrait la ligne des TAGS d'une question passée en paramètre };

    \node[text width=45.3969mm,below=0.75cm of a3.center,anchor=center,align=left] (a4) {\textbf{ D. } \texttt{aleaQ}     };
    \node[below=0.75cm of b3.center,anchor=center] (b4) {};
    \node[below=0.75cm of c3.center,anchor=center] (c4) {};
    \draw[thick,outer sep=8mm] (b4) circle(0.5ex);
    \draw[thick,outer sep=8mm] (c4) circle(0.5ex);
    \node[below=0.75cm of d3.center,anchor=center,text width=118.03194mm,align = right] (d4) {\textbf{ 4. } Génère un document pdf d'examen à partir d'une banque de questions };

\draw[very thick] (b3) -- (c4);
\draw[very thick] (b2) -- (c3);
\draw[very thick] (b4) -- (c1);
\draw[very thick] (b1) -- (c2);

\end{tikzpicture}
}%
\question{Quel est le but du script \texttt{genExam} ?}\\[0.0cm]
\indent\textbf{A.} {\color{BrickRed}    Faire le café {\large\xmark} }\newline
\indent\textbf{B.} {\color{OliveGreen}  Un prétexte pour faire du bash {\large\cmark} }\newline
\indent\textbf{C.} {\color{BrickRed}    Permet de ne pas aller en cours et de lancer un script le jour de l'examen {\large\xmark} }\newline
\indent\textbf{D.} {\color{OliveGreen}  Générer un document pdf d'examen à partir d'une banque de questions {\large\cmark} }\newline
\,\clearpage
\section*{Corrigé (Résumé)}
\noindent\textbf{Q.1} \newline
\indent\resizebox{\linewidth}{!}{%
\begin{tikzpicture}

    \node[text width=45.3969mm,align=left] (a1) at (0,0) {\textbf{ A. } \texttt{getQ}      };
    \node (b1) at ($(a1.east)+(1,0)$) {};
    \node (c1) at ($(b1.east)+(2,0)$) {};
    \draw[thick,outer sep=8mm] (b1) circle(0.5ex);
    \draw[thick,outer sep=8mm] (c1) circle(0.5ex);
    \node[text width=118.03194mm,align = right] (d1) at ($(c1.east)+(6,0)$) {\textbf{ 1. } Mélange les propositions d'une question passée en paramètre  };

    \node[text width=45.3969mm,below=0.75cm of a1.center,anchor=center,align=left] (a2) {\textbf{ B. } \texttt{getTAGS}   };
    \node[below=0.75cm of b1.center,anchor=center] (b2) {};
    \node[below=0.75cm of c1.center,anchor=center] (c2) {};
    \draw[thick,outer sep=8mm] (b2) circle(0.5ex);
    \draw[thick,outer sep=8mm] (c2) circle(0.5ex);
    \node[below=0.75cm of d1.center,anchor=center,text width=118.03194mm,align = right] (d2) {\textbf{ 2. } Renvoie la chaine de la question passée en paramètre };

    \node[text width=45.3969mm,below=0.75cm of a2.center,anchor=center,align=left] (a3) {\textbf{ C. } \texttt{genExam}   };
    \node[below=0.75cm of b2.center,anchor=center] (b3) {};
    \node[below=0.75cm of c2.center,anchor=center] (c3) {};
    \draw[thick,outer sep=8mm] (b3) circle(0.5ex);
    \draw[thick,outer sep=8mm] (c3) circle(0.5ex);
    \node[below=0.75cm of d2.center,anchor=center,text width=118.03194mm,align = right] (d3) {\textbf{ 3. } Extrait la ligne des TAGS d'une question passée en paramètre };

    \node[text width=45.3969mm,below=0.75cm of a3.center,anchor=center,align=left] (a4) {\textbf{ D. } \texttt{aleaQ}     };
    \node[below=0.75cm of b3.center,anchor=center] (b4) {};
    \node[below=0.75cm of c3.center,anchor=center] (c4) {};
    \draw[thick,outer sep=8mm] (b4) circle(0.5ex);
    \draw[thick,outer sep=8mm] (c4) circle(0.5ex);
    \node[below=0.75cm of d3.center,anchor=center,text width=118.03194mm,align = right] (d4) {\textbf{ 4. } Génère un document pdf d'examen à partir d'une banque de questions };

\draw[very thick] (b3) -- (c4);
\draw[very thick] (b2) -- (c3);
\draw[very thick] (b4) -- (c1);
\draw[very thick] (b1) -- (c2);

\end{tikzpicture}
}%
\newline
\textbf{Q.2} B. D. \newline
\end{document}
